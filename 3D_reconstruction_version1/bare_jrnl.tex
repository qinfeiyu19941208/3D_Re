\documentclass[journal]{IEEEtran}

\ifCLASSINFOpdf
\else
\fi

%\hyphenation{op-tical net-works semi-conduc-tor}

\usepackage{graphicx}
\usepackage{epstopdf}
\usepackage{indentfirst}
\usepackage{cite}
\usepackage{verbatim}
\usepackage{amsmath}
\usepackage{multirow}
\usepackage{bm}
\usepackage{array}
\usepackage[colorlinks,
            linkcolor=red,
            anchorcolor=black,
            citecolor=green
            ]{hyperref}
            \makeatletter
\newcommand{\rmnum}[1]{\romannumeral #1}
\newcommand{\Rmnum}[1]{\expandafter\@slowromancap\romannumeral #1@}
\makeatother
%\usepackage[justification=centering]{caption}
\begin{document}

\title{Efficient 3D City Modelling from Large-scale Aerial Images by Deep Learning}
%
%
% author names and IEEE memberships
% note positions of commas and nonbreaking spaces ( ~ ) LaTeX will not break
% a structure at a ~ so this keeps an author's name from being broken across
% two lines.
% use \thanks{} to gain access to the first footnote area
% a separate \thanks must be used for each paragraph as LaTeX2e's \thanks
% was not built to handle multiple paragraphs
%

\author{Michael~Shell,~\IEEEmembership{Member,~IEEE,}
        John~Doe,~\IEEEmembership{Fellow,~OSA,}
        and~Jane~Doe,~\IEEEmembership{Life~Fellow,~IEEE}% <-this % stops a space
\thanks{M. Shell was with the Department
of Electrical and Computer Engineering, Georgia Institute of Technology, Atlanta,
GA, 30332 USA e-mail: (see http://www.michaelshell.org/contact.html).}% <-this % stops a space
\thanks{J. Doe and J. Doe are with Anonymous University.}% <-this % stops a space
\thanks{Manuscript received April 19, 2005; revised August 26, 2015.}}

% The paper headers
\markboth{Journal of \LaTeX\ Class Files,~Vol.~14, No.~8, August~2015}%
{Shell \MakeLowercase{\textit{et al.}}: Bare Demo of IEEEtran.cls for IEEE Journals}
\maketitle

\begin{abstract}
Extracting buildings from remote sensing images plays an important role in urban applications (e.g., urban planning and digital city). However, this task is quite difficult due to great diversity of buildings and similarities between buildings and other categories. Recent approaches have attempted to harness the capabilities of deep learning techniques for building extraction. In this paper, we propose a robust system which can extract buildings from large-scale remote sensing images and build 3D models for extracted building areas. Learning low-level information of images becomes as important as learning high-level semantic information since buildings in remote images possess various scales and aspect ratios. So we propose a novel hierarchically fused fully convolutional network (HF-FCN). The proposed network generates the final prediction results in a fusion manner through making full use of the information extracted from each layer. Using modified VGG16 network, our method achieves state-of-the-art performance on several available remote sensing image datasets. In addition, we add the corresponding Digital Surface Model (DSM) map and extract the segmented building area to generate the point cloud of its roof. Then, based on the generated point cloud, the 3D modeling of buildings is implemented.
\end{abstract}
% Note that keywords are not normally used for peerreview papers.
\begin{IEEEkeywords}
building extraction, Hierarchically Fused Fully Convolutional Network (HF-FCN), 3D city modelling
\end{IEEEkeywords}
%\IEEEpeerreviewmaketitle
%
\section{Introduction}
\label{sec:intro}


\cxj{Is your goal reconstruction or building extraction? The introduction should explain the overall goal. What are the challenges for building modeling from remote sensing images? What kind of work has been done in the literature? Why do we focus on building detection? What are our contributions?}

\IEEEPARstart{B}{uilding} extraction, which aims to extract rooftop\footnote{Because the data sets used in our article are high altitude remote sensing images which could be considered as the top views of the ground. Therefore, we do not distinguish the concepts of buildings and rooftops in the subsequent description.} in a large-scale remote sensing image, remains one of the fundamental challenges have been studied for decades in the field of remote sensing. Moreover, auto extraction of building rooftops from aerial and satellite imagery is an important step in many applications, such as: urban planing, automated map making, 3D city modeling, updating geographical dataset and military reconnaissance. But, it is particularly difficult to extract rooftop at the pixel level for the following reasons:
\begin{itemize}
 \item Different density of buildings in the scene. A rural scene has low density but an urban scene has high density, with a suburban scene in between (medium density).
 \item Diverse shapes of the buildings. Buildings come in many shapes from simple rectangular blocks with flat roof to complex shapes with intricate roof shape.
 \item The quality of remote sensing images. Images vary in terms of contrast, resolution, and image principle \cite{IEEEexample:huertas1988detecting}.
\end{itemize}

 Several patches are shown in Fig.~\ref{fig:intro} \cxj{do not use the figure no directly. using ref{}..} , which illustrate the different challenges of building extraction task.


\begin{figure}
\includegraphics[width=8.7cm]{Figures/challenge.eps}
\caption{Examples of remote sensing patches with different kinds of challenges. (a) Shadow occlusion in green frame. (b) Low inter-class differences. (c) High intra class variance. (d) A lot of tiny buildings close to each other.}
\label{fig:intro}
\end{figure}


In the past decades, many researchers have made effort to extract buildings automatically.
At first, many simple knowledge-based methods were put forward by \cite{IEEEexample:huertas1988detecting}, \cite{IEEEexample:noronha2001detection}, \cite{IEEEexample:nosrati2009novel}, \cite{IEEEexample:izadi2012three}, \cite{IEEEexample:wang2015efficient}.
Their basic ideas are derived from prior knowledge that buildings are closed polygons made up of some straight lines.
Some others are energy-based methods including the variational level set evolution, improved snake model and graph cut \cite{IEEEexample:cote2013automatic}, \cite{IEEEexample:peng2005improved}, \cite{IEEEexample:sirmacek2009urban}. Due to early methods depend too much on prior and initialization. It does not apply to building extraction of complex scene.


In recent years, with the development of machine learning, many techniques via machine learning are gradually introduced into the remote sensing domain.
At first, some shallow networks were proposed for multiple geographic object extraction\cite{IEEEexample:mnih2013machine}, \cite{IEEEexample:saito2016multiple}, \cite{IEEEexample:alshehhi2017simultaneous},\cite{IEEEexample:zhao2017contextually}. Since methods use patches for segmentation, they are inefficient and inaccurate for the pixel-wise segmentation task.
Further, with the growth of computer power, deep learning developed rapidly and brought into the field of remote sensing. Some researchers tried deep learing for aerial images classification and semantic pixel labelling~\cite{IEEEexample:paisitkriangkrai2015effective}, \cite{IEEEexample:liu2017dense}, \cite{IEEEexample:audebert2017deep}, \cite{IEEEexample:kampffmeyer2017urban}, \cite{IEEEexample:he2017multi}. Unfortunately, owing to ignoring the hierarchical information extracted by the network, they could not deal with the case that scenes of close-packed buildings well.



To fix the above problems, a relatively simple, but very effective fusion operation is proposed in this work. And it could be combined into a general CNN architecture easily for building extraction. 
Differ from above mentioned methods, we take full advantages of the low-level appearance information as well as high-level semantic information by the novel fusion operation in a way of stage by stage.
Inspired by FCN\cite{IEEEexample:Long_2015_CVPR} whose output is in the same resolution of input, we propose a noval hierarchically fused FCN, named HF-FCN for buildings pixel-wise classification.
Differ from the traditional FCN, a set of hierarchical fusion operations are used to fuse the intra layer information and inter layer information respectively which improve the performance of FCN greatly.
And numerous experiments conducted on three remote sensing image datasets all obtain fairly good results.
Further, we extend our work to the field of 3D modeling as the part of building detection. It is easily intergrated into the pipeline of building reconstruction and achieve good performance.
Our technical contributions are:
%
\begin{enumerate}
	\item A effective hierarchical fusion operation which is specially designed for multi-scale building extraction is proposed. Combining with a general FCN, a novel network is presented, named HF-FCN that can deal with the problems of different sizes, diverse appearance and mutual occlusion of buildings and etc.
	\item HF-FCN is an end-to-end network that does not need any post processing. And the approach is significantly computationally efficient than existing techniques. Besides, the overall accuracy based on HF-FCN exceeds the state-of-art algorithms.
\end{enumerate}

The remainder of this paper is organized as follows. Sec.~\ref{Sec:RelatedWork} sums up the related works in the past.
In Sec.~\ref{Sec:HF-FCN}, we introduce the fusion operation and architecture of HF-FCN. The training loss are also presented.
And in Sec.~\ref{Sec:exp}, a brief description of the dataset used for our task is provided. HF-FCN training strategies, details and its evaluation metrics are also described.
In Sec.~\ref{Sec:Res}, we display and analysis the experimental results.
Extension in 3D building modeling are presented in Sec.~\ref{sec:app}.
Finally, the conclusion is discussed in Sec.~\ref{Sec:Con}.

\section{Related Work}
\label{Sec:RelatedWork}


Building extraction is one of the most fundamental problems in remote sensing domain, which has been studied for nearly 30 years. 
As time goes by, many research achievements have sprung up. 
We roughly divide these methods into three groups: one is based on the shape prior, another is based on the energy function and machine learning third. 
Here we briefly review some representative methods that have evolved in the past decades in the different groups respectively.
\cxj{More related work on city modeling/urban modeling. Maybe facade modeling.}



During early days, methods are mainly based on the hypothesis of prior knowledge.
Huertas and Nevatia \cite{IEEEexample:huertas1988detecting} assumed that buildings are rectangular or composed of rectangular components.
Based on this, the approach detected lines and corners, traced object boundaries and used shadows to verify. 
Later, a system \cite{IEEEexample:noronha2001detection} for building detection and modelling was proposed with the assumption that the roofs were flat or symmetrical and walls were vertical. 
Using known ground height and detected rooftop, the reconstructed models could be soon obtained. 
Further, Noronha and Nosrati \cite{IEEEexample:nosrati2009novel} transformed the line and intersection points of the image into a graph presentation, and turned the problem of polygon finding into the one that finding loops in the graph. 
However, it was still estimated on assumption that the buildings are polygonal. 
In addition, Izadi and Saeedi\cite{IEEEexample:izadi2012three} presented a complete system for building detection and modelling. 
In the stage of building detection, a tree consisting of intersection points of lines was created and refined based on the found hypotheses. 
The sun azimuth and elevation angles were used to estimate the height with existing shadows afterwards. 
As the height of buildings estimated, the three-dimensional polygonal building models were built. 
In recent years, very high resolution (VHR) optical satellite imagery could be obtained easily. 
Hence, Wang et al~\cite{IEEEexample:wang2015efficient} proposed an efficient method for automatic rectangular building extraction from VHR remote sensing images by detecting line segments and grouping lines based on path integrity and closed contour search.
\cxj{If there are two authors, say A and B proposed.... if more than two authors, say A et al. ..}


The aforementioned shape-based methods have a good performance in rural scenes with low density of buildings. 
Neverthless, there are several limitations of these methods.
First, the shape-based methods inherently limited to handle buildings of arbitrary shapes.
Second, they may failed to deal with complicated cases, for instance, buildings are close to each other, which thereby is hard to adapt to today's applications. 
Third, the algorithms using shadows to verify corners and estimate height are greatly limited to obvious shadows and sparse building environment.


Later, several energy-based methods in image segmentation domain have been applied in automatic rooftop extraction. 
Cote and Saeedi\cite{IEEEexample:cote2013automatic} employed corner detection as an initial estimate of the roof, and then refined with level set evolution. 
Peng et al~\cite{IEEEexample:peng2005improved} proposed an approach that segments remote sensing images into high objects, ground and shadow regions, with further refined by an improved snake model. 
The urban-region-detection problems were casted as one of multiple subgraph matching by Sirmacek and Unsalan\cite{IEEEexample:sirmacek2009urban}. 
They considered each SIFT keypoint as a vertex, neighborhood between vertexs as edge of the graph and formulated the problem of building detection in terms of graph cut.


Over the past decade, CNNs have achieved great success in the field of computer vision. 
There are significant amount of efforts on semantic pixel-level classification for extraction buildings in remote sensing. 
A shallow patch-based network was proposed by Mnih \cite{IEEEexample:mnih2013machine} which has only five layers with a 64 by 64 aerial patch as input. 
And the output of the network was processed by conditional random fields (CRFs). 
Afterwards, Satio et al. \cite{IEEEexample:saito2016multiple} applied two major strategies to improve the performance of the network. 
One was a channel-wise inhibited softmax (CIS) for getting a multi-label prediction result, the other was model averaging with spatial displacement (MA) for enhancing the prediction result. Alshehhi et al. \cite{IEEEexample:alshehhi2017simultaneous} also adjusted the architecture of network proposed by Mnih through changing the kernel size of convolutional layers and replacing the fully connection layer of the last layer with the average pooling layer. Alternative post-processing strategies such as CRFs and multi-scales were used to improve the final prediction results. Some methods took advantage of the feature extraction capability of CNNs to generate feature descriptions of patches. Paisitkriangkrai et al. \cite{IEEEexample:paisitkriangkrai2015effective} made use of both the CNN and hand-craft extracted features, which were combined together to generate predicted labels of each patch. They also used CRFs as post-processing to get a sound result. Zhao et al. \cite{IEEEexample:zhao2017contextually} proposed a method using edge information of VHR to guide semantic segmentation. Unlike \cite{IEEEexample:paisitkriangkrai2015effective}, \cite{IEEEexample:he2017multi} put forward a multi-label pixelwise classification method using the feature vector extracted by a CNN to train a Support Vector Machine (SVM) for classification.


More recently, Long et al. \cite{IEEEexample:Long_2015_CVPR} illustrated that Fully Convolutional Networks (FCN) could better handle the problem of multi-label pixel-wise classification. By up-sampling, final predicted result could be the same resolution of the input. Liu et al. \cite{IEEEexample:liu2017dense} did a further research on the formulation proposed by Paisitkriangkrai \cite{IEEEexample:paisitkriangkrai2015effective} but used FCN as the branch of CNN and applied a higher-order CRFs as post-processing. Unlike traditional CRFs, the label consistency for the pixels within the same segment were enforced by higher-order CRFs. In order to reduce the information loss during pooling stage, SegNet \cite{IEEEexample:badrinarayanan2017segnet} delivered pooling indices computed in the max-pooling to the decoder. It eliminated the need of learning during the up-sample stage while achieving good segmentation performance. The SegNet architecture was used by Audebert et al. \cite{IEEEexample:audebert2017deep} for semantic labeling of remote sensing and got better prediction results compared to the traditional methods. Later, Kampffmeyer at al. \cite{IEEEexample:kampffmeyer2017urban} proposed a novel idea that using CNN with missing data for urban land cover classification. The idea came from a modality hallucination architecture proposed by Hoffman et al. \cite{IEEEexample:hoffman2016learning} which learned with side information during training stage.


In the filed of computer vision, the FCNs \cite{IEEEexample:Long_2015_CVPR} were introduced as a powerful method for semantic segmentation and have achieved great performance. But, along with the deepening of network, the feature maps with lower resolution which causes the segmentation accuracy decline. In order to weaken the influence caused by pooling, Chen et al.\cite{IEEEexample:chen2016deeplab} proposed a atrous convolution which enlarged the receptive field and reduced the number of pooling layers at the same time. Vemulapalli et al.\cite{IEEEexample:vemulapalli2016gaussian} later extended the Deeplab \cite{IEEEexample:chen2016deeplab} with a pairwise network and proposed a Gaussian Conditional Random Field Network for more continuous segmentation results. Afterwards, with the advent of the powerful networks such as ResNet\cite{IEEEexample:he2016deep}, GoogLeNet\cite{IEEEexample:szegedy2015going} and their variants \cite{IEEEexample:szegedy2016rethinking}\cite{IEEEexample:szegedy2017inception}\cite{IEEEexample:xie2017aggregated}, a large amount of literature made use of these networks as their backbone for semantic segmentation. Zhao et al.\cite{IEEEexample:zhao2017contextually} recently developed a pyramid pooling module following the ResNet\cite{IEEEexample:he2016deep} to get multi-scale feature maps and connected these feature maps with those which before pyramid pooling to create the final prediction. Zuo et al.\cite{IEEEexample:zuo2016hf} described a hierarchically fused fully convolutional network, which combined the feature maps from each group of VGG16 Net to generate the final prediction. In this paper, we extend the work of\cite{IEEEexample:zuo2016hf} to explore the effect of different layers of features on the final result. And comparing with other mainstream semantic segmentation networks to prove our method more suitable to the building detection task.


\cxj{Need a paragraph to discuss recent semantic segmentation networks. }

\cxj{Also cite our accv paper and describe the relationship/difference of this journal paper with it.}

Although above-mentioned CNN-based models have exceeded the traditional methods significantly, all of them lost important hierarchical features encoded in the CNNs. They usually apply the CNN features from the last layer to get a segmentation result. It may omit tiny objects during the process of pooling, and could not handle the situation when the size of buildings have great difference in distribution. Aiming at this case, a hierarchical fusion operation is proposed to combine features extracted from each convolutional layer to capture various information of input images. We will describle the details of our idea below.

\section{Hierarchically Fused Fully Convolutional Network}
\label{Sec:HF-FCN}
In this section, we introduce a novel operation for feature fusion, named hierarchical fusion operation and apply it to the common networks, VGG16 Net and ResNet. The overview diagram in Fig.~\ref{fig:Fusion-Operation} shows where the fusion operations take effect and how they work. Different from other networks for semantic segmentation, we apply the fusion operation twice to integrate information gradually.
Our network consists of three parts. Part 1 is a backbone network whose role is to extract the features at different levels.
In theory, arbitrary feature extraction network is applicable to the Part 1.
The second part is a process of feature fusion in the first stage, which fuses the feature maps generated from each convolutional(conv) layer.
Besides, Part 3 is feature fusion in the second stage.
In the second stage of the fusion process, we take full advantage of the information extracted from the second part by learning the connection weights between upsampled feature maps.

\begin{figure}
\centering
\includegraphics[width=8.5cm]{Figures/Fusion_Operation.eps}
\caption{The first line shows the overview of our network. The second row shows the details of fusion operation. The one on the left is a case where the input is equal to the output and the one on the right is the case of the input not equal to the output.}
\label{fig:Fusion-Operation}
\end{figure}
\cxj{Put overview here. Explain the main components of our methods.}
\subsection{Network Architecture}
 Here, we illustrate our main idea using the VGG16 network as our backbone network, which have been proven to have better performance in experiments.
 Some modifications are made to apply to our building extraction task including removing its fc layers and last pooling layer. The reasons of these changes are
 1) The fc layer generates a fair number of parameters and takes up too much memory.
 2) The existence of fc layer limits the size of input image.
 3) After the last pooling layer, the resolution of the feature map is reduced to 1/32 of the input, which is too small to building extraction task.
 The details of our network using VGG16 Net as backbone network are shown in Fig.~\ref{fig:network_architecture}. The Level 1 in Fig.~\ref{fig:network_architecture} is a trimmed VGG16 Net which regards as our Part 1, backbone network.
 \cxj{Where do you define F1\_1? Is it the layer of VGG16 or HF-FCN?}
 %

In order to leverage the information extracted from different layers, we add the fusion branches on the backbone network.
The branches between Level 1 and Level 2 in Fig.~\ref{fig:network_architecture} form the second part of our network.
The idea is similar to getting the response of scale functions of images when looking for the SIFT feature points. Unlike the feature descriptor of SIFT, we use the concatenated conv layers as our feature extractor.
After getting the responses of different scale functions, the biggest response is selected between adjacent scales of each feature point.
The selecting process is determined by weights learned from fusion operations in our network.
From the perspective of neural networks, the fusion operations in first stage play a role of both feature compression and semantic information fusion in the same levels. They extracts the information from different scales of receptive field as well as diverse levels of semantics.
In addition, the whole weights are learned from the network automatically indicating that network studies the connection relationship among feature maps of same resolution.
\cxj{Fig 1 is the introduction figure.}
With the growing of the receptive field, the detailed information is captured by each conv layer from fine-grained to coarser while the semantic information captured from low level to high level.
For the task of rooftop extraction, not only the details of the appearance of the buildings captured by shallow layers is needed, but also the line and corner extracted by middle layers and the high-level semantics which mainly come from deep layers are needed.
Therefore, we extract various kinds of information by applying the fusion operations to the whole conv layers.
The upsampled feature maps from different conv layers are shown in Fig.~\ref{fig:feature_maps}.
The U1\_1 in Fig.~\ref{fig:feature_maps}(b) means the upsampled feature maps from F1\_1 which are feature maps generated from conv1\_1 with small receptive field extracts low-level features like edges.
In Fig.~\ref{fig:feature_maps}(c), the U1\_2 looks like an over-segmentation which groups pixels with similar color or texture into a subregion.
In the U2\_1, as Fig.~\ref{fig:feature_maps}(d) shows, shape information is augmented.
From the U3\_3, we can see that regions with significantly varying appearance are merged into an integrated building by considering high-level features.
In U4\_3 and U5\_3, more semantic information of rooftop is got, which can distinguish the rooftop and the roads with similar color and deal with the problem caused by shadow.
%

\begin{figure*}
\centering
\includegraphics[width=18cm,height=9.5cm]{Figures/network_architecture.eps}
\centering
\caption{Overall architecture of HF-FCN. The input of HF-FCN could be 3, 4 or 5 channels for RGB, DSM, nDSM. The backbone network is VGG16 network which contribute to the Level 1. The F1\_1 in Level 1 indicates the feature maps generated by conv1\_1. In Level 2, 13 upsampled feature maps are cropped to the same size of input. $\times$2 next to the fusion operation means 2 times of upper sampling. U1\_1 in Level 2 means the upsampled feature map of F1\_1, and so forth. \cxj{What is the different between our network with U-Net or other FCN networks?} }
\label{fig:network_architecture}
\end{figure*}

After getting the upsampled feature maps, we fuse them into a final prediction. This is the third part of our network.
Since all the upsampled feature maps are fused, it is expected to achieve a boost in rooftop segmentation which is shown in Fig.~\ref{fig:feature_maps}(h).
In this part, the fusion operation plays a role of feature weighting.
Our intention is learning a group of parameters to combine the upsampled feature maps which is similar to a process of feature selection.
The expression of the formula is as follows:
\begin{equation}
    \label{fature_selection}
    \ y(i,j)=\sum_{n=1}^{N}w_{n}U_{n}(i,j)
\end{equation}
where $y(i,j)$ is a point on the output, ${N}$ is the number of upsampled feature maps and ${U_{n}}$ is a upsampled feature map.


\begin{figure}
\centering
\includegraphics[width=8.7cm]{Figures/feature_maps.eps}
\caption{(a) Input aerial image. (b-g) Feature maps of U1\_1, U1\_2, U2\_2, U3\_3, U4\_3, U5\_3, respectively. (h) Predicted label map.}
\label{fig:feature_maps}
\end{figure}


\subsection{Network Training}

The ground truth $M$ in our dataset is labeled by 0 or 1 to indicate whether a pixel belongs to a roof or not. \cxj{only roof? or part of the building including facades?}
When a remote sensing image ${X}$ is inputted into the network, the output is a prediction probability map $P(X;W)$ of roof, where $W$ denotes all the parameters that learned by HF-FCN. Each pixel value in $P(X_{i};W)$ means the probability of this pixel belongs to rooftop.
We use the sigmoid cross-entropy loss function formulated as
\begin{small}
\begin{equation}
     \label{loss}
     \ L(W)\! =\! -\frac{1}{\vert I\vert}\sum_{i=1}^{\vert I \vert}\lbrack{\tilde{m}_i \log{P(X_{i};W)}\!+\!(1\!-\!\tilde{m}_i)\log(1\!-\!P(X_{i};W)}\rbrack,
\end{equation}
\end{small}
where $\tilde{m}_i$ is label of $X_{i}$, ${\vert I\vert}$ is the number of pixels in the input image ${X}$.

\section{Experiments}
\label{Sec:exp}

To verify the effectiveness of the proposed network, extensive experiments have been conducted on three remote sensing datasets. In this section, the experimental setup is described including details of datasets, training settings of HF-FCN and different criterion for evaluation.


\subsection{Dataset Description}

\paragraph{Massachusetts dataset}
%
Massachusetts dataset consists of 151 aerial images of the Boston area which covers roughly 340 square kilometers.
The resolution of each image is $1500\times 1500$ with the spacial resolution of 1 meter per pixel. And the images are composed of red, green and blue channels.
This dataset is built by Mnih while ground-truth is produced by Saito et al.
The dataset is split into three parts,  a training set of 137 images, a test set of 10 images and validation set of 4 images.
To train the network, we create a set of image tiles for training and validation by sliding a ${256\times256}$ window with 64 stride from right to left, top to bottom. The detailed description is shown in Table \ref{table:dataset-composition}.

\paragraph{Vaihingen dataset}
%
Vaihingen dataset is captured over Vaihingen which is a relatively small village with many detached buildings and small multi story buildings in Germany.
This dataset contains 16 labeled images whose spacial resolution is 9cm per pixel.
It consists of near infra-red, red, green, blue imagery with corresponding normalized digital surface models (nDSMs) and row DSMs. The dataset is divided into training set, validation set, and test set which have 11 images, 2 images, and 3 images respectively. The same crop operations are done as the Massachusetts dataset.

\paragraph{Potsdam dataset}
%
In the Potsdam dataset, there are 24 labeled images whose ground sampling distance is 5cm.
This dataset shows a typical historic city with large building blocks. In order to grasp the global information of the building, the spacial resolution of the original image is reduced from 6000$\times$6000 to 1500$\times$1500.
Each image in this dataset contains 5-channel information: red, green, yellow, DSM and nDSM.
We split the dataset into training, validation and test sets in a proportion of $7:2:1$.



Data augmentation is made on the Vaihingen dataset and the Potsdam dataset.
One reason is that methods using dataset ${a)}$ do not extend the data.
Hence, to make a fair comparison with other methods, we also do not extend it.
Another reason is that the data quantity of dataset ${b)}$ and ${c)}$ is not enough which may lead to inadequate training.
Therefore, some measures of data augmentation are made in dataset ${b)}$ and ${c)}$ including data rotation and mirror flipping.
Components of the datasets are listed in Table~\ref{table:dataset-composition}.
Meanwhile, some sampled patches of dataset ${a)}$, ${b)}$, ${c)}$ are shown in Fig.~\ref{fig:dataset_sample}\cxj{5}.


\begin{figure}
\centering
\includegraphics[width=8cm]{Figures/datasets.eps}
\caption{Sample patches on the three datasets  (a) Massachusetts dataset (b) Vaihingen dataset (c) Potsdam dataset}
\label{fig:dataset_sample}
\end{figure}

\begin{table}
 \centering
 \caption{Composition of dataset}
 \label{table:dataset-composition}
 \begin{tabular}{c|ccc}
\hline
& Masschusetts & Vaihingen & Potsdam\\  \hline
Labeled images & 151& 16 &24\\ \hline
GSD & 1m & 9cm & 5cm\\ \hline
Bands & R,G,B & IR,R,G,DSM & IR,R,G,B,DSM\\ \hline
%Training set & \tabincell{c}{1,3,5,7,13,\\17,21,23,26,32,37} &\tabincell{c}{2\_10,3\_10,3\_11,3\_12,\\4\_11,4\_12,5\_10,5\_12,\\6\_8,6\_8,6\_10,6\_11,\\6\_12,7\_7,7\_9,7\_11,7\_12}\\ \hline
Training images &137 & 11 & 17\\ \hline
Training patches&75938 &115088 &85000\\ \hline
Training patch size& 256$\times$256 & 256$\times$256 & 256$\times$256\\ \hline
Validation images & 4 & 3 & 4\\ \hline
validation patches &2500 & 28376 &25000 \\\hline
Validation patch size & 256$\times$256 & 256$\times$256 & 256$\times$256\\ \hline
%Test set & 15,30 &2\_12,6\_7,7\_8 \\ \hline
Test images & 10 & 2 & 3\\ \hline
\end{tabular}
\end {table}



\subsection{Training Settings}
HF-FCN is trained on dataset ${a)}$ firstly owing to large amounts of training data. The pre-trained model of VGG16 Net and ResNet are used to finetune our HF-FCN. We use the stochastic gradient descent algorithm with the learning rate divided by 10 for each 8000 iterations to train our network. The drop-out ratio is set to 0.5 which avoids overfitting. When the HF-FCN converges on the dataset ${a)}$, we transfer it to the other datasets. All experiments in this paper are performed using the deep learning framework Caffe and trained on a single NVIDIA Titan 12GB GPU. Besides, the hyper-parameters are listed in Table~\ref{table:Train-Parameter} \cxj{\Rmnum{3}}.

\begin{table}
\centering
\caption {Parameters for Network Training}
\label{table:Train-Parameter}
\begin{tabular}{c|c|c|c}
\hline
&Massachusetts &Vaihigen &Potsdam\\  \hline
%input size & 256$\times$256 &256$\times$256 \\
mini-batch size & 18& 15 & 15 \\
initial learning rate & $10^{-5}$ & $10^{-6}$ & $10^{-5}$\\
test\_interval&1000 & 1000 &1000\\
%type &SGD &SGD &SGD\\
training iteration & 10000 & 10000& 10000\\
momentum & 0.9 & 0.9 & 0.9\\
clip\_gradients & 16000& 10000 & 10000\\
weight\_decay & 0.02& 0.005 & 0.005\\ \hline
\end{tabular}
\end{table}

\subsection{Evaluation Metrics}
Several evaluation metrics are adopted in our work. For dataset ${a)}$, the most common metrics are correctness (precision) and completeness (recall).
The stantdard ($\rho$=0) and relaxed ($\rho$=3) precision and recall scores are used to evaluate the prediction results. Here the relaxed precision means the predicted pixels are within $\rho$ pixels of a true pixel while the relaxed recall is the true pixels are within $\rho$ pixels of a predicted pixel. Moreover, the time cost is used to measure the efficiency of our HF-FCN. For dataset ${b)}$ and ${c)}$, we use correctness, completeness and F1 score as evaluation metrics.


\begin{equation}
 {completeness} = \frac{TP}{TP+FN},
\end{equation}
\begin{equation}
{correctness} = \frac{TP}{TP+FP},
\end{equation}
\begin{equation}
{F1\_score}= 2\cdot\frac{completeness\cdot correctness}{completeness+correctness}
\end{equation}
%
where TP indicates the true positives, FP implies the false positive, TN means the true negatives and FN refers to the false negatives.

\section{Results and Discussion}
\label{Sec:Res}

In this section, the proposed method using dataset ${a)}$, ${b)}$, ${c)}$ are compared to the recent non-deep-learning algorithms, such as Minh-CNN\cite{IEEEexample:mnih2013machine}, Satio-multi~\cite{IEEEexample:saito2016multiple} and Context~\cite{IEEEexample:audebert2017deep}.
%
And it is also compared with some recent deep-learning based approaches, including FCN~\cite{IEEEexample:Long_2015_CVPR}, SegNet~\cite{IEEEexample:badrinarayanan2017segnet}, Deeplab~\cite{IEEEexample:chen2016deeplab} and U-Net~\cite{IEEEexample:ronneberger2015u}.\cxj{What else?}
Moreover, for HF-FCN itself, we expect to investigate which kind of information extracted from feature extractors and how the effects of extracted information on the final prediction.
Thus, some upsampled feature maps ${\left\{U_k\right\}}$ are presented.
And several variants of HF-FCN which combine different up-sampling feature maps from Part 2 are proposed.
They are shown in Fig.~\ref{fig:feature_maps} and Fig.~\ref{fig:Variants}, respectively.
In addition, we also tried different feature extractor networks (Part 1), VGG16 Net and ResNet. The details are shown below.
\cxj{What about to change the backbone network?}

\begin{figure}[t]
\begin{center}
\includegraphics[width=7.8cm]{Figures/vairants.eps}
\caption{FCN and HF-FCN variants. The feature maps generated from final group are fused into a coarse result, which is HF-FCN16s. The variant called HF-FCN8s concatenates the feature maps from the last 2 groups with the same fusion operation, and so on.}
\label{fig:Variants}
\end{center}
\end{figure}

\subsection{Massachusetts dataset}
On the Massachusetts dataset, our method is compared to both the non-deep-learning algorithms and deep-learning based approaches. Table~\ref{table:Mass-results}\cxj{4} present the quantitative analysis. A standard and relaxed precision and recall are amply to make a comparison.
From the result, our method shows obvious superiority in terms of speed and precision. When comparing with Satio\-multi\-MA\&CIS~\cite{IEEEexample:saito2016multiple}, the standard and relaxed recall of our method are $5.5\%$ and $1.3\%$ higher than it. Meanwhile, the time cost is reduced from 67.84s to 1.07s and the speed is promoted about 63 times.
These significant improvements demonstrate that HF-FCN achieves better performance in effectiveness and efficiency.

Extensive comparisons are made between HF-FCN and other mainstream methods in semantic segmentation domain. The quantitative and visual results are shown in Table~\ref{table:Mass-results} and Fig.~\ref{fig:Mass-visi-result}, respectively. 
Compared with U-Net~\cite{IEEEexample:ronneberger2015u}, the speed is promoted about 3 times and recall is a little higher. 
From the visual results, our method preserve the details and integrity of the building better than others.
\begin{table}
\centering
\caption {Correctness at breakeven of HF-FCN v.s. \cite{IEEEexample:mnih2013machine}\cite{IEEEexample:saito2016multiple}\cite{IEEEexample:alshehhi2017simultaneous}\cite{IEEEexample:Long_2015_CVPR}\cite{IEEEexample:badrinarayanan2017segnet}
\cite{IEEEexample:ronneberger2015u}\cite{IEEEexample:chen2016deeplab}on Massachusetts test set. Cost time is computed in the same computer with a single NVIDIA Titan 12GB GPU}
\label{table:Mass-results}
\begin{tabular}{cccc}
\hline
&Recall ($\rho$ = 3)&Recall ($\rho$ = 0)&Time (s)\\
\hline
Mnih-CNN \cite{IEEEexample:mnih2013machine}&0.9271&0.7661&8.70\\
Mnih-CNN+CRF\cite{IEEEexample:mnih2013machine} &0.9282&0.7638&26.60\\
Satio-multi-MA \cite{IEEEexample:saito2016multiple}&0.9503&0.7873&67.72\\
Satio-multi-MA\&CIS \cite{IEEEexample:saito2016multiple}&0.9509&0.7872&67.84\\
Alshehhi-GAP+seg \cite{IEEEexample:alshehhi2017simultaneous}&0.955&{--}&{--} \\ \hline
FCN\_4s\cite{IEEEexample:Long_2015_CVPR}&0.839&0.6147&4.20\\
SegNet\cite{IEEEexample:badrinarayanan2017segnet}&0.7710&0.5675&2.39\\
U-Net\cite{IEEEexample:ronneberger2015u}& $\bm{0.9638}$& $\bm{0.8357}$& 3.165\\
DeepLab\_V2\cite{IEEEexample:chen2016deeplab}&0.9620&0.7575&$\bm{1.89}$\\ \hline
HF-FCN(VGG16 Net)&0.9643& $\bm{0.8424}$ &1.07\\
HF-FCN(VGG+data aug)&$\bm{0.9650}$&0.8357&1.38\\
HF-FCN(ResNet)&0.9588&0.8175&2.42\\
HF-FCN16s &0.9330&0.7233&0.85\\
HF-FCN8s &0.9643&0.8171&0.93\\
HF-FCN4s &0.9632&0.8394&$\bm{0.99}$\\
\hline
\end{tabular}
\end{table}

In addition, to explore which kinds of information extracted by hierarchical fusion operation in Part 2.
Some upsampled feature maps ${\left\{U_1,U_2,U_3,U_7,U_{10},U_{13}\right\}}$ are shown in Fig.~\ref{fig:feature_maps}.
The U1\_1 (${U_1}$) in Fig.~\ref{fig:feature_maps}(b) means the upsampled feature maps from F1\_1 (${F_1}$) which are feature maps generated from conv1\_1 in VGG16 Net.
Due to small receptive field of conv1\_1 and conv1\_2, they extract low-level features like edges.
And the U1\_2 (${U_2}$) looks like an over-segmentation which groups pixels with similar color or texture into a subregion.
With the deepening of the network, in the U2\_1 (${U_3}$), as Fig.~\ref{fig:feature_maps}(d) shows, shape information is augmented.
And from the U3\_3 (${U_7}$), we can see that regions with significantly varying appearance are merged into an integrated building by considering high-level features.
In U4\_3 (${U_{10}}$) and U5\_3 ($U_{13}$), more semantic information of rooftop is got, which can distinguish the rooftop and the roads with similar color and deal with the problem caused by shadow.
The final prediction results are shown in Fig.~\ref{fig:feature_maps}(h).

Secondly, to explore the effects of the feature maps generated from each feature extract stage ${\left\{F_k\right\}}$ on the final result, variants of HF-FCN which are counterpart of FCN are designed.
Fig.~\ref{fig:Variants} shows the constrast diagram of variants of FCN and HF-FCN.
Unlike FCN, a fusion operation rather than summation are leveraged to build our HF-FCN 16s, 8s and 4s.
The PR curves, prediction results and quantitative results of HF-FCN variants are shown in Fig.~\ref{fig:Mass-variants-PR}, Fig.~\ref{fig:Mass-variants-visi} and Table~\ref{table:Mass-results} respectively. \cxj{Figure 8, Figure 9 and Table \Rmnum{5}}.
From the disgrams, we can get the following conclusions easily. 
First, The prediction result obtained from the last layer gets a coarse result, which loses much of location information that are mainly encoded in the shallow feature maps. Second, the largest gap presented between HF-FCN16s and HF-FCN8s about 9{\%} in recall rates, it may suggest that the most information supplement to the HF-FCN is got in middle layers. Third, the PR curves of HF-FCN4s and HF-FCN almost coincide. It illustrates the low-level information has little effect on the prediction results. Forth, with the addition of the shallow feature map, the network is more distinct for the segmentation of tiny buildings, which solves the problem of easy adhesion to adjacent buildings. 
Since, all the feature maps contained useful hierarchical information that is critical to the final prediction.

\begin{figure}
\begin{center}
\includegraphics[width=8.7cm]{Figures/feature_maps.eps}
\caption{(a) Input aerial image. (b-g) Feature maps of U1\_1, U1\_2, U2\_2, U3\_3, U4\_3, U5\_3, respectively. (h) Predicted label map. All the images are normalized to the range of ${0-255}$.}
\label{fig:feature_maps}
\end{center}
\end{figure}


In the end, we want to prove that our fusion operations learn the connections between feature maps. The connection weights of F1\_1, F4\_1 and Part 3 are shown in Fig.~\ref{fig:Mass-weights}.
The weights are not the same, which means that fusion operations have effect on feature combination.
From the Fig.~\ref{fig:Mass-weights} (a) to Fig.~\ref{fig:Mass-weights} (c), the range of weights increases gradually. And from the Fig.~\ref{fig:Mass-weights} (c), we can arrive at the conclusion that the different layers have virous effects on the final result.
For example, the U1\_1 has little effect on the prediction while the U3\_2 and U4\_3 have bigger role on the final prediction.
It also in accordance with our experimental results that middle layers provide more information.
\cxj{Weight for what? to fuse feature map? The distribution does not make too much sense. }

\begin{figure}
\vspace{-0.2cm}
\setlength{\abovecaptionskip}{-0cm}
\setlength{\belowcaptionskip}{-1.5cm}  
\centering
\includegraphics[width=8.7cm]{Figures/HF-FCN-variant-PR.eps}
\caption{The relaxed precision-recall curves from HF-FCN variants with two slack paramters. The biggest gap occurs between HF-FCN16s and HF-FCN8s, which indicates the most additional information coming from middle layers.}
\label{fig:Mass-variants-PR}
\end{figure}

\begin{figure}
\vspace{-0cm}
\setlength{\abovecaptionskip}{-0cm}  
\setlength{\belowcaptionskip}{-1.5cm}  
\centering
\includegraphics[width=8.7cm]{Figures/HF-FCN_variants_result.eps}
\caption{Prediction results of HF-FCN, HF-FCN4s, HF-FCN8s and HF-FCN16s. The yellow box shows the continuous refinement of the tiny buildings. The red and blue boxes show the mutual promotion and contradiction between different layers.}
\label{fig:Mass-variants-visi}
\end{figure}

\begin{figure*}
\vspace{-0.5cm}
\setlength{\abovecaptionskip}{-0cm}
\setlength{\belowcaptionskip}{-1cm}  
\centering
\includegraphics[width=17cm,height = 7cm]{Figures/mass_visi_result.eps}
\caption{(a) input images. (b) Results of Mnih-CNN+CRF. (c) Results of Satio\-multi\-MA\&CIS. (d) Results of FCN4s . (e) Results of SegNet. (f) Results of DeepLab\_V2. (g) Results of U-Net. (h) Our results. TP are shown in green, FP are shown in blue and FN are in red.}
\label{fig:Mass-visi-result}
\end{figure*}

\begin{figure}
\vspace{-0.2cm}
\setlength{\abovecaptionskip}{-0cm}
\setlength{\belowcaptionskip}{-1cm}  
\centering
\includegraphics[width=9cm]{Figures/weights.eps}
\caption{(a) is weights learned by F1\_1, (b) is weights learned by F4\_1, (c) is weights learned by Part 3.}
\label{fig:Mass-weights}
\end{figure}

\subsection{Vaihingen dataset}
On Vaihingen dataset, three experiments are undertaken to explore the effects of different inputs, diverse variants and various methods. We utilize three kinds of combinations of image channels as inputs. The inputs of the 3 channels are IR, R, G and adding the nDSM(normalized Digital Surface Model) as the forth channel. Based on it, DSM is added and made up 5-channel input. We use three standards to make a more comprehensive evaluation. The evaluation results are shown in Table~\ref{table:vaihigen-3-4-5in-comp}, which illustrates that 3-channel input performed better than the other. The Rec and Pre in Table~\ref{table:vaihigen-3-4-5in-comp} means the recall and precision of prediction results. And F1 indicates the F1\_score of results. The number in bold shows the best results in validation and test set. Corresponding visual results are shown in Fig.~\ref{fig:Vaihingen-3-4-5in}.

\cxj{Do you compare with others?}


\begin{table}[htbp]
\caption {Performance comparison of the results of different inputs and methods on Vaihigen data set. \cxj{What are the numbers in the Img column?}}
\label{table:vaihigen-3-4-5in-comp}
\centering
\begin{tabular}{p{0.5cm}<{\centering}|p{1.1cm}<{\centering}|p{1.1cm}<{\centering}|p{1.1cm}<{\centering}|p{1.1cm}<{\centering}}
\hline
%&\multirow{2}{*}{Img}&\multicolumn{3}{c}{3\_in: IR, R, G} &\multicolumn{3}{|c|}{4\_in: IR, R, G, nDSM}&\multicolumn{3}{c}{5\_in: IR, R, G, DSM, nDSM}\\
&&Pre&Rec&F1\\
\hline
\multirow{2}{*}{Val}&3\_in&0.939&0.894&$\bm{0.915}$\\
&4\_in&0.96&0.865&0.909\\
&5\_in&0.939&0.880&0.907\\
\hline
\multirow{2}{*}{Test}&3\_in&$\bm{0.919}$&$\bm{0.930}$&$\bm{0.925}$\\
&4\_in&0.907&0.872&0.888\\
&5\_in&0.858&0.900&0.878\\
\hline\hline
\multicolumn{2}{c|}{FCN\_4s\cite{IEEEexample:Long_2015_CVPR}}&{0.871}&$\bm{0.884}$&{0.878}\\
\multicolumn{2}{c|}{SegNet\cite{IEEEexample:badrinarayanan2017segnet}}&{0.917}&{0.861}&{0.887}\\
\multicolumn{2}{c|}{U-Net\cite{IEEEexample:ronneberger2015u}}&{0.848}&{0.737}&{0.789}\\
\multicolumn{2}{c|}{DeepLab\_V2\cite{IEEEexample:chen2016deeplab}}&$\bm{0.926}$&{0.881}&$\bm{0.903}$\\
\hline \hline
\multicolumn{2}{c|}{HF-FCN16s}&{0.886}&{0.854}&{0.870}\\
\multicolumn{2}{c|}{HF-FCN8s}&{0.911}&{0.864}&{0.887}\\
\multicolumn{2}{c|}{HF-FCN4s}&{0.910}&{0.861}&{0.885}\\
\hline
\end{tabular}
\end{table}

We compare with some other methods which use the same dataset and several deep learning methods. The detail comparison results using the same dataset are shown in Fig.~\ref{fig:Vaihingen-compared-others}. From a visual perspective, our method gets a much more refined roof region, both on continuity of labels and integrity of structural. The quantitative results of deep learning methods are shown in Table~\ref{table:vaihigen-3-4-5in-comp}. Compared to DeepLab\_V2\cite{IEEEexample:chen2016deeplab}, the F1\_score of our method improves 2.2{\%}.

The results of diverse variants are shown in Fig.~\ref{fig:Vaihingen-variants}. The HF-FCN\_1 in Fig.~\ref{fig:Vaihingen-variants} indicates that the last conv layer in Part 3 does not use the previous trained model to initialize. And HF-FCN means that the whole layers use the pre-trained model to initialize. From the curves, the performance of HF-FCN exceeds the variants and gets a excellent result. Additionally, using the pre-trained weights of Part 3 has a significance in the final results.

\begin{figure}
\vspace{-0.2cm}
\setlength{\abovecaptionskip}{-0cm}
\setlength{\belowcaptionskip}{-1cm}  
\centering
\includegraphics[width=8.7cm]{Figures/vaihingen_variants.eps}
\caption{Results of HF-FCN variants on Vaihingen dataset. (a) (b) shows the precision, recall and F1\_score of validation set and test set of Vaihingen dataset respectively.\cxj{Bigger font}}
\label{fig:Vaihingen-variants}
\end{figure}

\begin{figure}
\vspace{-0.2cm}
\setlength{\abovecaptionskip}{-0cm}
\setlength{\belowcaptionskip}{-1cm}  
\centering
\includegraphics[width=8.7cm]{Figures/Potsdam_variants.eps}
\caption{Results of HF-FCN variants on Potsdam dataset. (a) (b) shows the precision, recall and F1\_score of validation set and test set of Potsdam dataset respectively.}
\label{fig:Potsdam-variants}
\end{figure}

\begin{figure}
\vspace{-0.2cm}
\setlength{\abovecaptionskip}{-0cm}
\setlength{\belowcaptionskip}{-1cm}  
\centering
\includegraphics[width=8.7cm]{Figures/Vaihingen3_4_5in.eps}
\caption{Prediction results on Vaihingen dataset. (a) (b) (c) shows results of the 3-channel input, 4-channel input and 5-channel input of Vaihingen dataset respectively. Here, TP are shown in green, FP are shown in blue and FN are in red.}
\label{fig:Vaihingen-3-4-5in}
\end{figure}

\begin{figure}
\vspace{-0.2cm}
\setlength{\abovecaptionskip}{-0cm}
\setlength{\belowcaptionskip}{-2cm}  
\centering
\includegraphics[width=8.7cm]{Figures/Vaihingen_compared_results.eps}
\caption{Results of different methods. (a) is input image, (b)(d)(g) are results of \cite{IEEEexample:audebert2017deep}, (c) is result of \cite{IEEEexample:marmanis2016semantic}, (f) is result of \cite{IEEEexample:unknown}, (g) is our result. The blue and yellow frames show some details between these methods.}
\label{fig:Vaihingen-compared-others}
\end{figure}

\subsection{Potsdam dataset}
 The same experiments are implemented on Potsdam dataset, including effects of different channels of input, comparing with other methods and results of diverse variants of HF-FCN. Firstly, we utilize nDSM and IR information as extra inputs based on the RGB input. The specific quantitative evaluation and intuitive visual prediction results are shown in Table~\ref{table:Potsdam-3-4-5in-comp} and Fig.~\ref{fig:Potsdam-3-4-5in-visi}. In the validation process, the 4-channel input including RGB, nDSM gets better overall performance. Meanwhile, the 5-channel input including RGB, nDSM and IR seems perform better in the course of testing. From the visual results, the 5-channel input network gets lower error detection rate which is shown on the image with small blue areas. And from the 3-channel input to 5-channel input, the F1 score increases from 0.879 to 0.891 on the validation set and increases 0.031 on the test set. It indicate that the other information of geographical feature have a certain effect on the final result.

 We compare HF-FCN with other methods using the Potsdam dataset and several deep learning methods. Some qualitative results of methods using Potsdam datast are shown in Fig.~\ref{fig:Potsdam-compared-others}.
 From the figure, we can easily see that HF-FCN got more remarkable segmentation results. And edges and structure of buildings are preserved better. The results of deep learning methods are shown in Table~\ref{table:Potsdam-3-4-5in-comp}. From the Table, the HF-FCN achieves the best result. And the F1\_score far higher than the others.  

 As done on Vaihingen dataset, contrast experiments of HF-FCN variants are implemented. The performance curve of HF-FCN variants are shown in Fig.~\ref{fig:Potsdam-variants}.
 The HF-FCN\_1 in Fig.~\ref{fig:Potsdam-variants} indicates that the last conv layer in Part 3 does not use the previous trained model to initialize. And HF-FCN means that the whole layers use the pre-trained model to initialize. Initialization of parameters has a greater promotion on the final results.

\begin{table}[htbp]
\caption {Performance comparison of the results of different inputs and methods on Potsdam data set}
\label{table:Potsdam-3-4-5in-comp}
\centering
\begin{tabular}{p{0.5cm}<{\centering}|p{1.1cm}<{\centering}|p{1.1cm}<{\centering}|p{1.1cm}<{\centering}|p{1.1cm}<{\centering}}
\hline
%&\multirow{2}{*}{Img}&\multicolumn{3}{c}{3\_in: IR, R, G} &\multicolumn{3}{|c|}{4\_in: IR, R, G, nDSM}&\multicolumn{3}{c}{5\_in: IR, R, G, DSM, nDSM}\\
&&Pre&Rec&F1\\
\hline
\multirow{2}{*}{Val}&3\_in&0.937&0.851&0.879\\
&4\_in&0.937&$\bm{0.872}$&$\bm{0.894}$\\
&5\_in&$\bm{0.944}$&0.864&0.891\\
\hline
\multirow{2}{*}{Test}&3\_in&0.922&0.900&0.910\\
&4\_in&0.937&0.935&0.936\\
&5\_in&$\bm{0.940}$&$\bm{0.943}$&$\bm{0.941}$\\
\hline\hline
\multicolumn{2}{c|}{FCN\_4s\cite{IEEEexample:Long_2015_CVPR}}&{0.827}&{0.774}&{0.796}\\
\multicolumn{2}{c|}{SegNet\cite{IEEEexample:badrinarayanan2017segnet}}&{0.648}&{0.773}&{0.687}\\
\multicolumn{2}{c|}{U-Net\cite{IEEEexample:ronneberger2015u}}&$\bm{0.924}$&{0.705}&{0.799}\\
\multicolumn{2}{c|}{DeepLab\_V2\cite{IEEEexample:chen2016deeplab}}&{0.901}&$\bm{0.876}$&$\bm{0.887}$\\
\hline \hline
\multicolumn{2}{c|}{HF-FCN16s}&{0.849}&{0.857}&{0.851}\\
\multicolumn{2}{c|}{HF-FCN8s}&{0.846}&{0.831}&{0.838}\\
\multicolumn{2}{c|}{HF-FCN4s}&{0.851}&{0.839}&{0.844}\\
\hline
\end{tabular}
\end{table}

\begin{figure}
\vspace{-0.2cm}
\centering
\includegraphics[width=8.7cm]{Figures/Potsdam3_4_5in.eps}
\caption{Prediction results on potsdam dataset. (a) (b) (c) shows results of the 3-channel input, 4-channel input and 5-channel input of Vaihingen dataset respectively. Here, TP are shown in green, FP are shown in blue and FN are in red.}
\label{fig:Potsdam-3-4-5in-visi}
\end{figure}

\begin{figure}
\vspace{-0.2cm}
\centering
\includegraphics[width=8.7cm]{Figures/Potsdam_compared_results.eps}
\caption{Results of different methods. The second column is the results of using only the FCN with CIR(color-infrared image). Pairwise CRF fusion shows the result of fusing FCN-8s\_CIR with LiDAR data in a pairwise CRF. The results of using higher-order CRF\cite{IEEEexample:liu2017dense} as post processing are shown in third column. The last colunm shows our results.}
\label{fig:Potsdam-compared-others}
\end{figure}

\section{Conclusion}
\label{Sec:Con}
 In this paper, we propose a complete system for efficient 3D city modelling from large-scale aerial images via deep learning. A novel CNN architecture, HF-FCN, combines hierarchical semantic layers and multi-scale feature representation to implement final building detection. We design a fusion branch to fuse the feature map stage by stage. And the resulting HF-FCN method shows significant improvements over several previous methods. Distinct from the previous deeplearning based methods, we utilize the multi-scale inherent information within the CNN and get fine detail detection results. Unlike existing 3D  reconstruction methods, our proposed approach relies on the HF-FCN to efficiently extract the area of buildings to build the 3D models of the geospatial objects. Finally, our study suggests that even with the powerful semantic expressive ability of CNNs and their good robustness to scale, it is still critical to address multi-scale problems utilizing hierarchical feature maps encoded in CNNs.
 
\bibliographystyle{IEEEtran}
\bibliography{IEEEexample}
\end{document}



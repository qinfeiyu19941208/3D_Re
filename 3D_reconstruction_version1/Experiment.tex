\documentclass{article}
\usepackage{multirow}
\usepackage{bm}
\usepackage{graphicx}
\usepackage{geometry}
%\geometry{left=3.0cm,right=3.0cm,top=2.5cm,bottom=2.5cm}
\begin{document}
\newcommand{\tabincell}[2]{\begin{tabular}{@{}#1@{}}#2\end{tabular}}
\section{Experiments}
In this Section,we introduce our detailed implementation of building detection,and show the results of building detection.
\subsection{Dataset}
The overhead images used in this work comes from the ISPRS 2D Semantic Labelling Challenge Vaihigen and Potsdam.Vaihigen data consist of near infra-red(IR),red(R),green(G) imagery with corresponding digital surface models(DSMs).Potsdam data consist of near infra-red,red,green,blue imagery with corresponding normalized digital surface models(nDSMs) and row digital surface models(DSMs).Both areas cover urban scenes,while Vaihigen is a relative small village with many detached buildings and small multi story buildings,Potsdam shows a typical historic city with large building blocks,narrow streets and dense settlement structure.
Since images in the data set are too large to train in the network,we cut the training and validation images into256$\times$256 pixels patches.In addition,since the images provided by Potsdam data are too large,it may affect the final prediction result, we reduce the spatial resolution of the original resolution to 1/16.Component of the data sets are shown in Table 1.

\begin{table}[!h!b!p]
 \centering
 \caption{Composition of dataset}
 \begin{tabular}{|c|c|c|}
\hline
&Vaihigen &Potsdam\\  \hline
Labeled images &16 &24\\ \hline
GSD & 9cm & 5cm\\ \hline
Bands &IR,R,G,DSM &IR,R,G,B,DSM\\ \hline
Training set & \tabincell{c}{1,3,5,7,13,\\17,21,23,26,32,37} &\tabincell{c}{2\_10,3\_10,3\_11,3\_12,\\4\_11,4\_12,5\_10,5\_12,\\6\_8,6\_8,6\_10,6\_11,\\6\_12,7\_7,7\_9,7\_11,7\_12}\\ \hline
Training patches &115088 &85000\\ \hline
validation set & 11,28,34 &2\_11,4\_10,5\_11,7\_10\\ \hline
validation patches & 28376 &25000 \\\hline
Test set & 15,30 &2\_12,6\_7,7\_8 \\ \hline
\end{tabular}
\end {table}

\subsection{Training Settings}
The implementation of our network  based on the Caffe Library.Our network is initialized and fine-tuned by pre-trained HF-FCN model and trained in an end-to-end manner.HF-FCN is trained on the Massachusetts Building Dataset,which consists of 151 aerial images of the Boston area.The standard parameters used during training the networks are shown in Table 2.In order to avoid over-fitting, we set the drop-out ratio to 0.5.Because the network converged very fast,we trained only 20000 times,then chose the model which performed better both on the validation set and test set for building detection.Training parameters on two data sets are shown in Table 2.
\begin{table}
\centering
\caption {Parameters for Network Training}
\begin{tabular}{c|c|c}
\hline
&Vaihigen &Potsdam\\  \hline
input size & 256$\times$256 &256$\times$256 \\
mini-bachsize & 15 &15 \\
learning rate & 10$^-6$ & 10$^-5$\\
test\_interval& 1000 &1000\\
type & SGD &SGD\\
max\_iter& 40000& 40000\\
momentum&0.9 &0.9\\
clip\_gradients & 10000 &10000\\
weight\_decay&0.005 &0.005\\ \hline
\end{tabular}
\end{table}

\subsection {Results}
We applied three different information as input to compare its impact on our network performance.For the Vaihigen data set,we set up three different inputs,the 3-channel inputs are IR,R,G,the 4-channel inputs are IR,R,G,nDSMs,the 5-channel inputs are IR,R,G,DSMs,nDSMs.Table 3 shows the different prediction results.We adopted 3 different evaluation metrics,nameed precision,recall and F1-score.
\begin{table}[h!b!p]
\caption {Performance comparison of the results of different inputs on Vaihigen data set}
\begin{center}
\begin{tabular}{c|c|c|c|c|c|c|c|c|c|c}
\hline
&\multirow{2}{*}{Img}&\multicolumn{3}{c}{3\_in:IR,R,G} &\multicolumn{3}{|c|}{4\_in:IR,R,G,nDSMs}&\multicolumn{3}{c}{5\_in:IR,R,G,DSM,nDSM}\\
\cline{3-11}
&& Pre &Rec & F1 &Pre &Rec &F1&Pre &Rec &F1\\
\hline
\multirow{3}{*}{Val}&11&0.911&0.906&0.909&0.936&0.900&0.917&0.890&0.900&0.900\\
&28&0.94&0.875&0.906&0.96&0.792&0.868&0.952&0.823&0.883\\
&34&0.965&0.899&0.930&0.987&0.902&0.942&0.972&0.918&0.944\\
&Ave&0.939&0.894&$\bm{0.915}$&0.961&0.865&0.909&0.939&0.880&0.907\\
\hline
\multirow{2}{*}{Test}&15&0.918&0.930&0.924&0.883&0.917&0.9&0.833&0.931&0.88\\
&30&0.921&0.929&0.926&0.931&0.827&0.876&0.875&0.877&0.876\\
&Ave&0.919&0.930&$\bm{0.925}$&0.907&0.872&0.888&0.858&0.900&0.878\\
\hline
\end{tabular}
\end{center}
\end{table}
\\We did a similar experiment on the Potsdam data set,and compared the effects of different inputs on semantic segmentation.Unlike the Vaihigen dataset, on the Potsdam data set the 3-channel inputs are R,G,B, the 4-channel inputs are IR, R,G,B and the 5-channel inputs are IR, R,G,B, nDSMs.Table 4 shows the different results.The evaluation metrics are the same as Vaihigen data set.
   \begin{table}[!h!b!p]
    \caption {HF-FCN semantic labelling results on Potsdam data set}
    \begin{center}
    \begin{tabular}{c|c|c|c|c|c|c|c|c|c|c}
     \hline
    &\multirow{2}{*}{Img}&\multicolumn{3}{c}{3\_in:RGB} &\multicolumn{3}{|c|}{4\_in:RGB,IR}&\multicolumn{3}{c}{5\_in:RGB,IR,nDSM}\\
     \cline{3-11}
    && Pre &Rec & F1 &Pre &Rec &F1&Pre &Rec &F1\\
    \hline\hline
    \multirow{4}{*}{Val}&2\_11&0.917&0.950&0.933&0.917&0.978&0.946&0.934&0.976&0.954\\
    &4\_10&0.937&0.945&0.941&0.926&0.943&0.936&0.947&0.946&0.946\\
    &5\_11&0.930&0.972&0.950&0.959&0.975&0.966&0.956&0.977&0.967\\
    &7\_10&0.964&0.536&0.689&0.950&0.590&0.728&0.939&0.554&0.697\\
    \cline{2-11}
    &{Average}&0.937&0.851&0.879&0.937&$\bm{0.872}$&$\bm{0.894}$&$\bm{0.944}$&0.864&0.891\\
    \hline\hline
    \multirow{3}{*}{Test}&2\_12&0.897&0.868&0.882&0.920&0.959&0.939&0.944&0.965&0.955\\
    &6\_7&0.894&0.902&0.898&0.915&0.909&0.912&0.901&0.918&0.909\\
    &7\_8&0.975&0.929&0.951&0.977&0.950&0.957&0.976&0.946&0.960\\
    \cline{2-11}
    &{Average}&0.922&0.900&0.910&0.937&0.935&0.936&$\bm{0.940}$&$\bm{0.943}$&$\bm{0.941}$\\
    \hline\hline
   \end{tabular}
   \end{center}
   \end{table}
%\begin{figure}[h]
%\begin{center}
%\includegraphics[width=135mm]{vai_result.png}
%\caption{(a)input images.(b)Results of 3 channels inputs.(c)Results of 4 channels inputs.(d)Results of 5 channels inputs.Correct results(TP)are shown in green,false positive(FP)are shown in blue.false negative(FN)are shown in red.}
%\centering
%\label{fig:VaihigenResults}
%\end{center}
%\end{figure}
%\begin{figure}[h]
%\begin{center}
%\includegraphics[width=135mm]{Potsdam_result.png}
%\caption{(a)input images.(b)Results of 3 channels inputs.(c)Results of 4 channels inputs.(d)Results of 5 channels inputs.Correct results(TP)are shown in green,false positive(FP)are shown in blue.false negative(FN)are shown in red.}
%\label{fig:ComparedResults}
%\end{center}
%\end{figure}


\end{document}
